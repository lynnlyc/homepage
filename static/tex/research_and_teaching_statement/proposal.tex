% https://github.com/MCG-NKU/NSFC-LaTex
% https://www.overleaf.com/read/jydxqkkkskzp
% by Ming-Ming Cheng https://mmcheng.net
% 关于VsCode LaTeX的配置 https://www.cnblogs.com/ourweiguan/p/11785660.html

\documentclass[12pt]{article}


\usepackage[UTF8]{ctex}
\usepackage{nsfc}


%\usepackage{fontspec}
%\usepackage{xcolor}
%\defaultfontfeatures{Ligatures=TeX}




\newcommand{\cmm}[1]{\textcolor[rgb]{0,0.6,0}{CMM: #1}}
\newcommand{\todo}[1]{{\textcolor{red}{\bf [#1]}}}
\newcommand{\myPara}[1]{\paragraph{#1:}}

\graphicspath{{figures/}}


\begin{document}



%%%%%%%%% TITLE

\title{报告正文}

\maketitle

\ContentDes{(一)立项依据与研究内容(建议8000字以下):}


\NsfcSection{1}{项目的立项依据}{
(研究意义、国内外研究现状及发展动态分析,需结合科学研究发展趋势来论述科学意义;
或结合国民经济和社会发展中迫切需要解决的关键科技问题来论述其应用前景。
附主要参考文献目录);}




\subsection{研究意义}


\textbf{Windows下Tex Live最新版可以用pdfLatex快速编译
和标准模板相似度更高的文档。}
%
显著性源于视觉的独特性、不可预测性、稀缺性以及奇异性,并且是由颜色、梯度、边缘、
边界等图像属性所致。
视觉显著性和我们如何感知、处理视觉刺激紧密相关,
并且正在被包括认知心理学~\cite{Wolfe_attributesVisual},
神经生物学
和计算机视觉在内的多个学科进行研究。



\subsection{国内外相关工作}

关于人类视觉注意的理论假设人类视觉系统只详细地处理图像的某个局部,而不是整幅图像。
Treisman 和 Gelade~\cite{Treisman_featureIntegration}, Koch 和
Ullman~\cite{KochVisualAttention}的早期工作,以及Itti, Wolfe 等人的视觉注意理论提议将视觉注意机制分为两个阶段:
快速的、下意识的、自底向上的、数据驱动的显著性提取;
以及慢速的、任务依赖的、自顶向下的、目标驱动的显著性提取。



{
\bibliographystyle{ieee}
\bibliography{Cmm}
}


%%%%%%%%%%%%%%%%%%%%%%%%%%%%%%%%%%%%%%%%%%%%%%%%%
\NsfcSection{2}{项目的研究内容、研究目标,以及拟解决的关键科学问题}{
(此部分为重点阐述内容);}

\subsection{研究目标}
我们提出了基于全局对比度的显著性计算方法,即基于直方图对比度~(HC) 和基于空间信息增强的区域对比度~(RC)方法。
HC方法速度快,并且产生细节精确的结果,RC方法可以产生空间增强的高质量显著性图像,但与此同时具有相对较低的计算效率。
我们在国际上现有最大的公开数据集上测试了我们的方法,并与之前已有八种最好的其它方法进行了比较。
实验结果表明,我们提出的方法在正确率和召回率上都明显优于其它方法,并且简单而高效。


\begin{figure}[ht]
	\centering
    \begin{overpic}[width=0.6\columnwidth]{teaser.jpg}
    \end{overpic}
    \caption{给定输入图像(上),可以通过全局对比度分析得到高分辨率的视觉显著性图(中)。
         这种视觉显著性图可以进一步被用来获取感兴趣物体区域(下)。
    }\label{fig:teaser}
\end{figure}

\subsection{研究内容}

在未来的工作中,我们计划研究包含空间关系且保留详细细节的显著性图像的高效计算算法,
并且希望研究能够处理具有复杂纹理的背景图像的显著性检测算法,
以克服我们现有算法在处理这类情况中存在的缺陷。
最后,我们还希望显著性图像的检测过程中进一步考虑人脸、对称性等高级因素。
我们相信显著性图像可以应用于高效物体检测\cite{han_unsupervised},
可靠图像分类,鲁棒的图像景物分析~\cite{ChengZMHH10},
并提高图像检索效果。


\subsection{拟解决的关键科学问题}



\NsfcSection{3}{拟采取的研究方案及可行性分析}{
(包括研究方法、技术路线、实验手段、关键技术等说明);}


\subsection{拟采取的技术路线}


\subsection{可行性分析}


\NsfcSection{4}{本项目的特色与创新之处;}{}

\NsfcSection{5}{年度研究计划及预期研究结果}{
(包括拟组织的重要学术交流活动、国际合作与交流计划等)。}

\subsection{年度研究计划}


\subsection{预期研究成果}


%%%%%%%%%%%%%%%%%%%%%%%%%%%%%%%%%%%%%%%%%%%%%%%%%
\ContentDes{(二)研究基础与工作条件}


\NsfcSection{1}{研究基础}{
(与本项目相关的研究工作积累和已取得的研究工作成绩);}

\subsection{工作基础1}
\subsection{工作基础2}



\subsection{研究工作获奖}


\NsfcSection{2}{工作条件}{
(包括已具备的实验条件,尚缺少的实验条件和拟解决的途径,
包括利用国家实验室、
国家重点实验室和部门重点实验室等研究基地的计划与落实情况);}


\myPara{经费和硬件条件方面}我们


\myPara{人员方面}我们

\myPara{国内外合作方面} 我们


\NsfcSection{3}{正在承担的与本项目相关的科研项目情况}{
(申请人和项目组主要参与者正在承担的与本项目相关的科研项目情况,
包括国家自然科学基金的项目和国家其他科技计划项目,
要注明项目的名称和编号、经费来源、起止年月、与本项目的关系及负责的内容等);}


%%%%%%%%%%%%%%%%%%%%%%%%%%%%%%%%%%%%%%%%%%%%%%%%%
\ContentDes{(三) 其他需要说明的问题}



\NsfcSection{1}{}{
申请人同年申请不同类型的国家自然科学基金项目情况
(列明同年申请的其他项目的项目类型、项目名称信息,
并说明与本项目之间的区别与联系)。}


\NsfcSection{2}{}{
具有高级专业技术职务(职称)的申请人或者主要参与者是否存在
同年申请或者参与申请国家自然科学基金项目的单位不一致的情况;
如存在上述情况,列明所涉及人员的姓名,
申请或参与申请的其他项目的项目类型、项目名称、单位名称、
上述人员在该项目中是申请人还是参与者,并说明单位不一致原因。}



\NsfcSection{3}{}{
具有高级专业技术职务(职称)的申请人或者主要参与者是否具有
高级专业技术职务(职称)的申请人或者主要参与者是否存在与正
在承担的国家自然科学基金项目的单位不一致的情况;如存在上述情况,
列明所涉及人员的姓名,正在承担项目的批准号、项目类型、项目名称、
单位名称、起止年月,并说明单位不一致原因。}


\NsfcSection{4}{}{其他。}

无


\end{document}
