% https://github.com/MCG-NKU/NSFC-LaTex
% https://www.overleaf.com/read/jydxqkkkskzp
% by Ming-Ming Cheng https://mmcheng.net
% 关于VsCode LaTeX的配置 https://www.cnblogs.com/ourweiguan/p/11785660.html

\documentclass[12pt]{article}


\usepackage[UTF8]{ctex}
\usepackage{nsfc}


%\usepackage{fontspec}
%\usepackage{xcolor}
%\defaultfontfeatures{Ligatures=TeX}




\newcommand{\lyc}[1]{\textcolor[rgb]{0,0.6,0}{LYC: #1}}
\newcommand{\todo}[1]{{\textcolor{red}{\bf [#1]}}}
\newcommand{\myPara}[1]{\paragraph{#1:}}

\graphicspath{{figures/}}


\begin{document}



%%%%%%%%% TITLE

\title{教学陈述}

\maketitle

\begin{center} {李元春} \end{center}


我深信教学相长、温故知新的道理,参与教学不仅是身为一个教师的职责,同时也是一个反思和凝练研究工作的过程。在过去的求学和工作过程中,我有过一些参与教学和指导的经历,在自己获得提升的同时,获得了帮助他人成功的成就感。未来我也非常期待作为一名教师和导师,继续提升自己并对更多人产生正面的影响。

\section{教学和指导经历}

在攻读博士期间,我曾多次作为助教,协助导师参与各项课程的教学,如《计算机系统导论》、《操作系统》、《编译原理》等。我的工作内容包括部分课件的制作、专题的讲授、作业批改和答疑、讨论课的组织等等。这些经历一方面使我熟悉了一门课程背后的运作模式、锻炼了总结和讲授的能力,同时也通过与老师同学的讨论和互动,学到了很多新的知识。

在读期间,我还多次作为国际非盈利安全研究组织 The Honeynet Project 的成员,担任 Google Summer of Code 开源项目的导师,指导国内外学生参与开源项目。从这些这些经历中,我学习了如何面试和挑选合适的学生,体会到如何帮助学生指定项目计划,管理项目的进度,从而推进项目的成功。同时,在这些项目中,我还帮助学生与国际同行展开合作和交流,从而获得更大的影响力。

毕业之后,我在微软亚洲研究院工作期间指导过多个实习生参与科研,其中大部分的实习时间为六个月以上。对于每个实习生,我带领其完成一个研究项目从无到有的全过程,帮助他们学习如何寻找研究问题、如何制定解决方案、如何进行学术演讲、以及如何撰写学术论文等等。大部分实习生都在我的指导下完成了一个完整的项目,有些还发表了高水平的学术论文。在帮助他们成功的同时,我也获得了莫大的成就感。

我还作为讲师参与了由微软亚洲研究院开设的《AI System》在线课程,在其中担任安全与隐私部分的讲师,在这门课程中,我将领域内的前沿研究进行归纳总结,形成结构化的课程。同时,这段经历也让我体会了在线授课与传统授课方式在课程设计、课堂互动等方面的异同。


\section{预计可讲授课程}

\subsection{《计算机系统》、《编译原理》、《软件工程》}
我在本科学习期间就对这些课程就有着浓厚的兴趣,在取得了优异的成绩的同时,直接促使我选择了相关方向的实验室和导师展开了科研工作。博士在读期间,又多次担任这些基础课程的助教,进一步加深了对相关课程内容的理解,并熟悉了课程背后的教案准备、作业批改、试题设计等内容。这些基础课程是历久弥新的,对我未来的研究工作也有温故知新的作用,因此我也可以讲授这些基础课程。

\subsection{《智能软件工程》}
卡内基梅隆大学开设有两门与智能软件工程相关的课程,分别是 AI4SE \cite{cmu:ai4se} 和 SE4AI \cite{cmu:se4ai},分别讲解如何用AI技术解决软件工程中的经典问题(如代码分析、需求分析、测试等)和如何用软件分析技术解决AI模型相关的重要问题(例如模型鲁棒性测试和验证、决策过程分析与解释、模型调试等)。我对这两个方向都较为熟悉,可以讲授相关课程。

\subsection{《人工智能的安全、隐私与伦理》}
神经网络的安全隐私及伦理问题是如今热门的研究方向之一,已有很多大学开设了相关课程 \cite{ucb:trustworthy,ucb:fairness}。我在微软亚洲研究院开源课程《人工智能系统》 \cite{microsoft:ai-system}中担任安全与隐私章节的讲师,也对该方向做过调研和整理,可以讲授相关课程。




{
\bibliographystyle{ieee_fullname}
\bibliography{lyc}
}


\end{document}
